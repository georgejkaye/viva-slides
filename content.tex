\begin{frame}
    \frametitle{The big picture}



\end{frame}

\begin{frame}
    \frametitle{Contributions}



    \begin{center}
        \begin{minipage}{0.45\textwidth}
            \centering
            \textbf{Semantics of Digital Circuits}
        \end{minipage}
        \begin{minipage}{0.45\textwidth}
            \centering
            \textbf{Graph Rewriting for Digital Circuits}
        \end{minipage}
    \end{center}

\end{frame}

\begin{frame}
    \frametitle{Chapter 3: Syntax}

    \textbf{Previously:} syntax and semantics confusingly intermingled with
    layers of quotients and equations

    \pause

    \textbf{Now:} syntax of circuits defined before considering any behaviour

    \pause

    \vspace{1em}

    \begin{center}
        \begin{minipage}{0.4\textwidth}
            \centering
            {\LARGE\(\ccircsigma\)}

            \vspace{1em}

            \alert{Combinational} circuits

            (functions)
        \end{minipage}
        \pause
        {\LARGE\(\hookrightarrow\)}
        \begin{minipage}{0.4\textwidth}
            \centering
            {\LARGE\(\scircsigma\)}

            \vspace{1em}

            \alert{Sequential} circuits

            (stateful)
        \end{minipage}
    \end{center}

\end{frame}

\begin{frame}
    \frametitle{Chapter 4: Denotational semantics}

    \pause

    \textbf{Previously:} denotations of circuits only considered informally

    \pause

    \textbf{Now:} assign behaviour to circuit morphisms in terms of
    \alert{causal, monotone, and finitely specified stream functions} using
    a construction via \alert{Mealy machines}

    \pause

    \vspace{1em}

    \begin{center}
        \begin{minipage}{0.15\textwidth}
            \centering
            {\LARGE\(\scircsigma\)}
        \end{minipage}
        \pause
        \(\xrightarrow{\circuittomealyi}\)
        \begin{minipage}{0.25\textwidth}
            \centering
            {\LARGE\(\mealyi\)}

            \vspace{0.5em}

            \alert{Monotone} Mealy machines
        \end{minipage}
        \pause
        \(\xrightarrow{\mealytocircuiti}\)
        \begin{minipage}{0.375\textwidth}
            \centering
            {\LARGE\(\streami\)}

            \vspace{0.5em}

            \alert{Causal, monotone, finitely specified} stream functions
        \end{minipage}
    \end{center}

\end{frame}

\begin{frame}
    \frametitle{Chapter 4: Denotational semantics}

    \begin{center}
        \LARGE
        \(\scircsigma \xrightarrow{\circuittostreami} \streami\)
    \end{center}

    \[
        \circuittostreami[
            \dsptikzfig{strings/category/f}[box=f,colour=seq]
        ]
        =
        \circuittostreami[
            \dsptikzfig{strings/category/f}[box=g,colour=seq]
        ]
        \quad
        \Rightarrow
        \quad
        \dsptikzfig{strings/category/f}[box=f,colour=seq]
        \extequivi
        \dsptikzfig{strings/category/f}[box=g,colour=seq]
    \]


    \begin{center}
        \LARGE
        \alert{Denotational equivalence}

        \vspace{1em}

        \(\scircsigmai \cong \streami\)
    \end{center}


\end{frame}

\begin{frame}
    \frametitle{Chapter 5: Operational semantics}

    \textbf{Previously:} reduction strategy for \alert{closed} circuits with
    \alert{delay-guarded} feedback

    \textbf{Now:} reduction strategy for any \alert{open} circuit using novel
    rule for \alert{unrolling non-delay-guarded feedback}

    \[
        \dsptikzfig{circuits/instant-feedback/equation-lhs}[box=f]
        \reduction
        \dsptikzfig{circuits/instant-feedback/concrete-unfolding}[box=f]
    \]

    \vspace{0.5em}

    \scalebox{0.9}{
        \(
        \dsptikzfig{circuits/productivity/productive-goal-lhs}[box=f]
        \reduction
        \dsptikzfig{circuits/productivity/mealy-form-applied}[core=\hat{f}]
        \reduction
        \dsptikzfig{circuits/productivity/mealy-form-instant-registers}[core=\hat{f}]
        \reduction
        \dsptikzfig{circuits/productivity/productive-goal-rhs}[box=g]
        \)
    }

\end{frame}

\begin{frame}
    \frametitle{Chapter 5: Operational semantics}

    \begin{center}
        \begin{minipage}{0.55\textwidth}
            \centering
            \dsptikzfig{strings/category/f}[box=f,colour=seq]
            and
            \dsptikzfig{strings/category/f}[box=g,colour=seq]
            produce the same

            \vspace{0.5em}

            outputs using the productive
            strategy
        \end{minipage}
        \quad
        \(\Rightarrow\)
        \begin{minipage}{0.3\textwidth}
            \centering
            \(
            \dsptikzfig{strings/category/f}[box=f,colour=seq]
            \sim_{\interpretation}
            \dsptikzfig{strings/category/f}[box=g,colour=seq]
            \)
        \end{minipage}
    \end{center}

    \vspace{0.25em}

    \begin{center}
        \LARGE
        \alert{Observational equivalence}

        \vspace{0.5em}

        \(
        \dsptikzfig{strings/category/f}[box=f,colour=seq]
        \sim_{\interpretation}
        \dsptikzfig{strings/category/f}[box=g,colour=seq]
        \Leftrightarrow
        \dsptikzfig{strings/category/f}[box=f,colour=seq]
        \extequivi
        \dsptikzfig{strings/category/f}[box=g,colour=seq]
        \)

        \vspace{0.5em}

        \(\scircsigmai \cong \streami\)
    \end{center}


\end{frame}

\begin{frame}
    \frametitle{Chapter 6: Algebraic semantics}



\end{frame}

\begin{frame}
    \frametitle{Chapter 7: Potential applications}



\end{frame}

\begin{frame}
    \frametitle{Chapter 8: String diagrams as hypergraphs}



\end{frame}

\begin{frame}
    \frametitle{Chapter 9: Graph rewriting}



\end{frame}

\begin{frame}
    \frametitle{Chapter 10: Applications of graph rewriting}



\end{frame}
