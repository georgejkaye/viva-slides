\begin{frame}
    \frametitle{The big picture}



\end{frame}

\begin{frame}
    \frametitle{Contributions}



    \begin{center}
        \begin{minipage}{0.45\textwidth}
            \centering
            \textbf{Semantics of Digital Circuits}
        \end{minipage}
        \begin{minipage}{0.45\textwidth}
            \centering
            \textbf{Graph Rewriting for Digital Circuits}
        \end{minipage}
    \end{center}

\end{frame}

\begin{frame}
    \frametitle{Chapter 3: Syntax}

    In previous work syntax and semantics intermingled

    \pause

    We define two \alert{syntactic categories}:

    \pause

    \(\ccircsigma\) of \emph{combinational circuits}

    \pause

    \(\scircsigma\) of \emph{sequential circuits}

    \pause

    These are circuits in which circuit morphisms are \alert{constructed}

    \pause

    They have \alert{no behaviour} associated with them

\end{frame}

\begin{frame}
    \frametitle{Chapter 4: Denotational semantics}

    Previously behaviour of circuit morphisms only considered informally



    We assign \alert{behaviour} to circuit morphisms in terms of
    \alert{causal, monotone, and finitely specified stream functions}


    \[
        \circuittostreami[
        \iltikzfig{strings/category/f}[box=f,colour=seq]
        ]
        =
        \circuittostreami[
        \iltikzfig{strings/category/f}[box=g,colour=seq]
        ]
        \quad
        \Rightarrow
        \quad
        \iltikzfig{strings/category/f}[box=f,colour=seq]
        \extequivi
        \iltikzfig{strings/category/f}[box=g,colour=seq]
    \]

    \Large
    \alert{Denotational equivalence}

\end{frame}

\begin{frame}
    \frametitle{Chapter 5: Operational semantics}



\end{frame}

\begin{frame}
    \frametitle{Chapter 6: Algebraic semantics}



\end{frame}

\begin{frame}
    \frametitle{Chapter 7: Potential applications}



\end{frame}

\begin{frame}
    \frametitle{Chapter 8: String diagrams as hypergraphs}



\end{frame}

\begin{frame}
    \frametitle{Chapter 9: Graph rewriting}



\end{frame}

\begin{frame}
    \frametitle{Chapter 10: Applications of graph rewriting}



\end{frame}
